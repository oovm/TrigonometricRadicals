\chapter{n = 17 (高斯型)}\label{ch:cos17}

注意到 $3$ 是模 $17$ 的原根.

也就是说 $ω^3$ 是乘法模群 $(\mathbb{Z}/17\mathbb{Z})^{×}$ 的生成元.

其中

$$
\begin{aligned}
ω^{17} + 1& = 0\\
ω^{n} &= ω^{n\bmod 17}
\end{aligned}
$$

记 $σ_{i,j} = σ^{i}(ζ_{j})$.

\section{第零轨道}

不妨设

$$
\begin{aligned}
ζ_0
&=ω^{3^{0}}+ω^{3^{1}}+ω^{3^{2}}+ω^{3^{3}}+ω^{3^{4}}+ω^{3^{5}}+ω^{3^{6}}+ω^{3^{7}}+ω^{3^{8}}\\
&+ω^{3^{9}}+ω^{3^{10}}+ω^{3^{11}}+ω^{3^{12}}+ω^{3^{13}}+ω^{3^{14}}+ω^{3^{15}}+ω^{3^{16}}
\end{aligned}
$$

对这个东西的幂模 $17$, 得到第零轨道

$$
\begin{aligned}
σ_{0,0}
&=ω^{1}+ω^{3}+ω^{9}+ω^{10}+ω^{13}+ω^{5}+ω^{15}+ω^{11}\\
&+ω^{16}+ω^{14}+ω^{8}+ω^{7}+ω^{4}+ω^{12}+ω^{2}+ω^{6}\\
\end{aligned}
$$

根据单位根的性质

$$σ^{0}(ζ_{0})=-1$$

然后根据伽罗瓦扩张 

$$
(\mathbb{Z}/17\mathbb{Z})^{×}
{\,\mathop▷\limits^2\,} K_{4}
{\,\mathop▷\limits^2\,} K_{3}
{\,\mathop▷\limits^2\,} K_{2}
{\,\mathop▷\limits^2\,} K_{1}
{\,\mathop▷\limits^2\,} K_{0}
≅ \mathbb{Q}
$$

把其他几个轨道写出来

---

\section{第一轨道}


$$
ζ_1=ω^{3^{0}}+ω^{3^{2}}+ω^{3^{4}}+ω^{3^{6}}+ω^{3^{8}}+ω^{3^{10}}+ω^{3^{12}}+ω^{3^{14}}+ω^{3^{16}}
$$

$$
\begin{aligned}
σ^{0}(ζ_{1})&=ω^{1}+ω^{9}+ω^{13}+ω^{15}+ω^{16}+ω^{8}+ω^{4}+ω^{2}\\
σ^{1}(ζ_{1})&=ω^{3}+ω^{10}+ω^{5}+ω^{11}+ω^{14}+ω^{7}+ω^{12}+ω^{6}\\
\end{aligned}
$$

\section{第二轨道}


$$
ζ_{2}=ω^{3^{0}}+ω^{3^{4}}+ω^{3^{8}}+ω^{3^{12}}
$$

$$
\begin{aligned}
σ^{0}(ζ_{2})&=ω^{1}+ω^{13}+ω^{16}+ω^{4}\\
σ^{1}(ζ_{2})&=ω^{3}+ω^{5}+ω^{14}+ω^{12}\\
σ^{2}(ζ_{2})&=ω^{9}+ω^{15}+ω^{8}+ω^{2}\\
σ^{3}(ζ_{2})&=ω^{10}+ω^{11}+ω^{7}+ω^{6}
\end{aligned}
$$

\section{第三轨道}


$$
ζ_{3}=ω^{3^{0}}+ω^{3^{8}}
$$

$$
\begin{aligned}
σ^{0}(ζ_{3})&=ω^{1}+ω^{16}\\
σ^{1}(ζ_{3})&=ω^{3}+ω^{14}\\
σ^{2}(ζ_{3})&=ω^{9}+ω^{8}\\
σ^{3}(ζ_{3})&=ω^{10}+ω^{7}\\
σ^{4}(ζ_{3})&=ω^{13}+ω^{4}\\
σ^{5}(ζ_{3})&=ω^{5}+ω^{12}\\
σ^{6}(ζ_{3})&=ω^{15}+ω^{2}\\
σ^{7}(ζ_{3})&=ω^{11}+ω^{6}
\end{aligned}
$$

后面还有个第四轨道, 和这个题无关就不写了.

---

我们来研究第一轨道

$$
|K_1/\mathbb{Q}| = |K_1/K_0|=2
$$



考虑

$$
\begin{aligned}
σ_{0,1}+σ_{1,1}&=ω^{16}+ω^{15}+ω^{14}+ω^{13}+ω^{12}+ω^{11}+ω^{10}+ω^9\\
&+ω^8+ω^7+ω^6+ω^5+ω^4+ω^3+ω^2+ω\\
σ_{0,1}σ_{1,1}&=ω^{30}+ω^{29}+ω^{28}+3 ω^{27}+2 ω^{26}+2 ω^{25}\\
&+ω^{24}+3 ω^{23}+3 ω^{22}+3 ω^{21}+4 ω^{20}+4 ω^{19}+4 ω^{18}+4 ω^{16}\\
&+4 ω^{15}+4 ω^{14}+3 ω^{13}+3 ω^{12}+3 ω^{11}+ω^{10}+2 ω^9+2 ω^8\\
&+3 ω^7+ω^6+ω^5+ω^4\\
\end{aligned}
$$

模掉 $ω^{17}$ 得到

$$
\begin{aligned}
σ_{0,1}σ_{1,1}
&=4 ω^{16}+4 ω^{15}+4 ω^{14}+4 ω^{13}+4 ω^{12}+4 ω^{11}+4 ω^{10}+4 ω^9\\
&+4 ω^8+4 ω^7+4 ω^6+4 ω^5+4 ω^4+4 ω^3+4 ω^2+4 ω\\
\end{aligned}
$$

也就是说 

$$
\begin{aligned}
σ_{0,1}+σ_{1,1}&=σ_{0,0}\\
σ_{0,1}σ_{1,1}&=4σ_{0,0}
\end{aligned}
$$

因为这个是韦达定理的形式, 不妨记 

$$
V(σ_{0,1}, σ_{1,1}) = [σ_{0,0} , 4 σ_{0,0}]
$$

---

同样的方法研究第二轨道发现

$$
\begin{aligned}
V(σ_{0,2}, σ_{2,2})&=[σ_{0,1}, σ_{0,0}]\\
V(σ_{1,2}, σ_{3,2})&=[σ_{1,1}, σ_{0,0}]\\
\end{aligned}
$$

同理对于第三轨道有

$$
\begin{aligned}
V(σ_{0,3}, σ_{4,3})&=[σ_{0,2}, σ_{1,2}]\\
V(σ_{1,3}, σ_{5,3})&=[σ_{1,2}, σ_{2,2}]\\
V(σ_{2,3}, σ_{6,3})&=[σ_{2,2}, σ_{3,2}]\\
V(σ_{3,3}, σ_{7,3})&=[σ_{3,2}, σ_{0,2}]\\
\end{aligned}
$$

因此所有的根和轨道都能由二次根式所表示.


---

好的, 说了这么多, 一个三角函数都没看见啊, 和题目到底有什么关系呢?


这就要说到欧拉公式了, 建立了指数函数和三角函数之间的桥梁

$$
\cos x=\frac{1}{2}e^{-ix}+\frac{1}{2}e^{ix}
$$

$$
\begin{aligned}
\cos \left(\frac{2 π }{17}\right)&=\frac{1}{2}e^{-2 i π/17}+\frac{1}{2}e^{2 i π/17}=\frac{ω^{-1}+ω^1}{2}\\
\sin \left(\frac{π }{17}\right)&=\sqrt{\frac{1}{2}-\frac{1}{2} \cos \left(\frac{2 π }{17}\right)}=\sqrt{\frac{1}{2}-\frac{1}{4}σ_{0,3}}
\end{aligned}
$$

然后查上面的轨道表合成序列:

$$
σ_{0,3} 
←[σ_{0,2}, σ_{1,2}]
←[σ_{0,1}, σ_{1,1}, σ_{0,0}]
←[σ_{0,0}]
$$

所以最终结果一定有四重根式

然后解轨道表得到

$$
\begin{aligned}
σ_{0,0}&= -1\\
σ_{0,1}&= \frac{-1+\sqrt{17}}{2}\\
σ_{1,1}&= \frac{-1-\sqrt{17}}{2}\\
σ_{0,2}&= \frac{1}{2} \left(\frac{-1+\sqrt{17}}{2}+\sqrt{\frac{17-\sqrt{17}}{2}}\right)\\
σ_{1,2}&=\frac{1}{2} \left(\frac{-1-\sqrt{17}}{2}+\sqrt{\frac{17+\sqrt{17}}{2}}\right)\\
\end{aligned}
$$

于是最终解得

$$
\begin{aligned}
\cos\frac{2π}{17}&=+\frac{1}{8} \left(β+\sqrt[4]{17} \sqrt{β}+\sqrt{2} \sqrt{7+3 α -4 \sqrt[4]{17} \sqrt{α }+\sqrt{36-13 β}}\right) \\
\cos\frac{4π}{17}&=+\frac{1}{8} \left(β-\sqrt[4]{17} \sqrt{β}+\sqrt{2} \sqrt{7+3 α +4 \sqrt[4]{17} \sqrt{α }-\sqrt{36-13 β}}\right) \\
\cos\frac{6π}{17}&=-\frac{1}{8} \left(α -\sqrt[4]{17} \sqrt{α }-\sqrt{2} \sqrt{7-3 β+4 \sqrt[4]{17} \sqrt{β}-\sqrt{36+13 α }}\right) \\
\cos\frac{8π}{17}&=+\frac{1}{8} \left(β+\sqrt[4]{17} \sqrt{β}-\sqrt{2} \sqrt{7+3 α -4 \sqrt[4]{17} \sqrt{α }+\sqrt{36-13 β}}\right) \\
\cos\frac{10π}{17}&=-\frac{1}{8} \left(α -\sqrt[4]{17} \sqrt{α }+\sqrt{2} \sqrt{7-3 β+4 \sqrt[4]{17} \sqrt{β}-\sqrt{36+13 α }}\right) \\
\cos\frac{12π}{17}&=-\frac{1}{8} \left(α +\sqrt[4]{17} \sqrt{α }-\sqrt{2} \sqrt{7-3 β-4 \sqrt[4]{17} \sqrt{β}+\sqrt{36+13 α }}\right) \\
\cos\frac{14π}{17}&=-\frac{1}{8} \left(α +\sqrt[4]{17} \sqrt{α }+\sqrt{2} \sqrt{7-3 β-4 \sqrt[4]{17} \sqrt{β}+\sqrt{36+13 α }}\right) \\
\cos\frac{16π}{17}&=+\frac{1}{8} \left(β-\sqrt[4]{17} \sqrt{β}-\sqrt{2} \sqrt{7+3 α +4 \sqrt[4]{17} \sqrt{α }-\sqrt{36-13 β}}\right) \\
\end{aligned}
$$

where

$$
\begin{aligned}
α &= \frac{\sqrt{17}+1}{2}\\
β &= \frac{\sqrt{17}-1}{2}\\
\end{aligned}
$$

最后 

$$
\sin\frac{π}{17}=\frac{1}{4} \sqrt{8-\sqrt{2 \sqrt{17}-2 \sqrt{34-2 \sqrt{17}}+2 \sqrt{12 \sqrt{17}+16 \sqrt{2 \sqrt{17}+34}+2 \sqrt{34-2 \sqrt{17}}-2 \sqrt{578-34 \sqrt{17}}+68}+30}}
$$