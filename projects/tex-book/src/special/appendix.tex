\chapter{版本更新历史}

根据用户的反馈,我们不断修正和完善模板。由于 3.00 之前版本与现在版本差异非常大,在此不列出 3.00 之前的更新内容。

\datechange{2018/12/06}{版本 3.00 正式发布}

\begin{change}
  \item 删除 \lstinline{mathpazo} 数学字体选项。
  \item 添加邮箱命令 \lstinline{\mailto}。
  \item 修改英文字体为 \lstinline{newtx} 系列,另外大型操作符号维持 cm 字体。
  \item 中文字体改用 \lstinline{ctex} 宏包自动设置。
  \item 删除 \lstinline{xeCJK} 字体设置,原因是不同系统字体不方便统一。
  \item 定理换用 \lstinline{tcolobox} 宏包定义,并基本维持原有的定理样式,优化显示效果,支持跨页;定理类名字重命名,如 etheorem 改为 theorem 等等。
  \item 删去自定义的缩进命令 \lstinline{\Eindent}。
  \item 添加参考文献宏包 \lstinline{natbib}。
  \item 颜色名字重命名。
\end{change}

\nocite{*}
\printbibliography[heading=bibintoc, title=\ebibname]


\chapter{基本数学工具}


本附录包括了计量经济学中用到的一些基本数学,我们扼要论述了求和算子的各种性质,研究了线性和某些非线性方程的性质,并复习了比例和百分数。我们还介绍了一些在应用计量经济学中常见的特殊函数,包括二次函数和自然对数,前 4 节只要求基本的代数技巧,第 5 节则对微分学进行了简要回顾;虽然要理解本书的大部分内容,微积分并非必需,但在一些章末附录和第 3 篇某些高深专题中,我们还是用到了微积分。

\section{求和算子与描述统计量}

\textbf{求和算子} 是用以表达多个数求和运算的一个缩略符号,它在统计学和计量经济学分析中扮演着重要作用。如果 $\{x_i: i=1, 2, \ldots, n\}$ 表示 $n$ 个数的一个序列,那么我们就把这 $n$ 个数的和写为:

\begin{equation}
\sum_{i=1}^n x_i \equiv x_1 + x_2 +\cdots + x_n
\end{equation}
