\usepackage{amsmath}
\usepackage[utf8]{inputenc}
\usepackage[T1]{fontenc}

设 7 次单位根:




设

$$
\begin{aligned}
    ω&= e^{2πi/3}\\
    ω^n&=ω^{n\bmod 3}
\end{aligned}
$$

也就是说 $ω=\sqrt[3]{\frac{-1+i \sqrt{3}}{2}}$

另一方面 $\{\cos 20°, \cos 40°, \cos 80°, \cos 100°, \cos 140°, \cos 160°\}$ 正好是方程 64 x^6-96 x^4+36 x^2-3 = 0$ 的六个实根.

求解该方程并用三角恒等变换后整理可得:

$$
\begin{aligned}
    \cos 10°&=+\frac{i}{2} \left(ω^7-ω^2\right)\\
    \cos 20°&=-\frac{1}{2} \left(ω^5+ω^4\right)\\
    \cos 30°&=+\frac{\sqrt{3}}{2}\\
    \cos 40°&=+\frac{1}{2} \left(ω^8+ω\right)\\
    \cos 50°&=+\frac{i}{2} \left(ω^8-ω\right)\\
    \cos 60°&=+\frac{1}{2}\\
    \cos 70°&=+\frac{i}{2} \left(ω^5-ω^4\right)\\
    \cos 80°&=+\frac{1}{2} \left(ω^7+ω^2\right)\\
\end{aligned}
$$

进而可以求出 $\tan$ 值的表为:

$$
\begin{aligned}
    \tan 10°&=-i\cdot\frac{ω^5+1}{ω^5-1}\\
    \tan 20°&=-i\cdot\frac{ω-1}{ω+1}\\
    \tan 30°&=+\frac{\sqrt{3}}{3}\\
    \tan 40°&=+i\cdot\frac{ω^7-1}{ω^7+1}\\
    \tan 50°&=+\frac{i}{2} \left(ω^8-ω\right)\\
    \tan 60°&=+\sqrt{3}\\
    \tan 70°&=+\frac{i}{2} \left(ω^5-ω^4\right)\\
    \tan 80°&=+\frac{1}{2} \left(ω^7+ω^2\right)\\
\end{aligned}
$$
