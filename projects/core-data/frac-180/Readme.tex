\usepackage{amsmath}
\usepackage[utf8]{inputenc}
\usepackage[T1]{fontenc}


因为是共轭根, 省略相反数.

---

$$
\begin{aligned}
    P(x)
    &= 4 x^5-5 x^3+x\\
    &= \prod x(x±1)(2x±1)\\
\end{aligned}
$$

$$
\begin{aligned}
    \cos 0°   &= +1\\
    \cos 60°  &= +\frac{1}{2}\\
    \cos 90°  &= 0\\
    \cos 120° &= -\frac{1}{2}\\
    \cos 180° &= -1\\
\end{aligned}
$$

---

$$
\begin{aligned}
    P(x)
    &= 8 x^4-10 x^2+3\\
    &= \left(4 x^2-3\right) \left(2 x^2-1\right)\\
\end{aligned}
$$

$$
\begin{aligned}
    \cos 30°  &= +\frac{\sqrt{3}}{2}\\
    \cos 45°  &= +\frac{\sqrt{2}}{2}\\
    \cos 135° &= -\frac{\sqrt{2}}{2}\\
    \cos 150° &= -\frac{\sqrt{3}}{2}\\
\end{aligned}
$$

---

$$
\begin{aligned}
    P(x) &= \prod 4 x^2±2 x-1\\
\end{aligned}
$$

$$
\begin{aligned}
    \cos 36°  &= +\frac{\sqrt{5}+1}{4}\\
    \cos 72°  &= +\frac{\sqrt{5}-1}{4}\\
    \cos 108° &= -\frac{\sqrt{5}-1}{4}\\
    \cos 144° &= -\frac{\sqrt{5}+1}{4}\\
\end{aligned}
$$

---

$$
\begin{aligned}
    P(x) &= 16 x^4-16 x^2+1\\
\end{aligned}
$$

$$
\begin{aligned}
    \cos 15°  &= +\frac{\sqrt{6}+\sqrt{2}}{4}\\
    \cos 75°  &= +\frac{\sqrt{6}-\sqrt{2}}{4}\\
    \cos 105° &= -\frac{\sqrt{6}-\sqrt{2}}{4}\\
    \cos 165° &= -\frac{\sqrt{6}+\sqrt{2}}{4}\\
\end{aligned}
$$

---

$$
\begin{aligned}
    P(x) &= 16 x^4-20 x^2+5\\
\end{aligned}
$$

$$
\begin{aligned}
    \cos 18°  &= +\frac{1}{2} \sqrt{\frac{1}{2} \left(5+\sqrt{5}\right)}\\
    \cos 54°  &= +\frac{1}{2} \sqrt{\frac{1}{2} \left(5-\sqrt{5}\right)}\\
    \cos 126° &= -\frac{1}{2} \sqrt{\frac{1}{2} \left(5-\sqrt{5}\right)}\\
    \cos 162° &= -\frac{1}{2} \sqrt{\frac{1}{2} \left(5+\sqrt{5}\right)}\\
\end{aligned}
$$

---

$$
\begin{aligned}
    P(x) &= 16 x^4+8 x^3-16 x^2-8 x+1\\
    P(x) &= 16 x^4-8 x^3-16 x^2+8 x+1\\
\end{aligned}
$$

$$
\begin{aligned}
    \cos 12°  &= +\frac{1}{8} \left(\sqrt{30+6\sqrt{5}}+\sqrt{5}-1\right)\\
    \cos 84°  &= +\frac{1}{8} \left(\sqrt{30-6\sqrt{5}}-\sqrt{5}-1\right)\\
    \cos 132° &= -\frac{1}{8} \left(\sqrt{30-6\sqrt{5}}-\sqrt{5}-1\right)\\
    \cos 156° &= -\frac{1}{8} \left(\sqrt{30+6\sqrt{5}}+\sqrt{5}-1\right)\\
\end{aligned}
$$

$$
\begin{aligned}
    \cos 24°  &= +\frac{1}{8} \left(\sqrt{30-6\sqrt{5}}+\sqrt{5}+1\right)\\
    \cos 48°  &= +\frac{1}{8} \left(\sqrt{30+6\sqrt{5}}-\sqrt{5}+1\right)\\
    \cos 96°  &= -\frac{1}{8} \left(\sqrt{30+6\sqrt{5}}-\sqrt{5}+1\right)\\
    \cos 128° &= -\frac{1}{8} \left(\sqrt{30-6\sqrt{5}}+\sqrt{5}+1\right)\\
\end{aligned}
$$

---

$$
\begin{aligned}
    P(x) &= 256 x^8-448 x^6+224 x^4-32 x^2+1\\
\end{aligned}
$$

$$
\begin{aligned}
    \cos 6°   &= +\frac{1}{8} \left(\sqrt{10-2\sqrt{5}}+\sqrt{15}+\sqrt{3}\right)\\
    \cos 42°  &= +\frac{1}{8} \left(\sqrt{30-6\sqrt{5}}-\sqrt{5}-1\right)\\
    \cos 66°  &= -\frac{1}{8} \left(\sqrt{30+6\sqrt{5}}-\sqrt{5}+1\right)\\
    \cos 78°  &= -\frac{1}{8} \left(\sqrt{30-6\sqrt{5}}+\sqrt{5}+1\right)\\
    \cos 102° &= +\frac{1}{8} \left(\sqrt{30-6\sqrt{5}}+\sqrt{5}+1\right)\\
    \cos 114° &= +\frac{1}{8} \left(\sqrt{30+6\sqrt{5}}-\sqrt{5}+1\right)\\
    \cos 138° &= -\frac{1}{8} \left(\sqrt{30-6\sqrt{5}}-\sqrt{5}-1\right)\\
    \cos 174° &= -\frac{1}{8} \left(\sqrt{30+6\sqrt{5}}+\sqrt{5}-1\right)\\
\end{aligned}
$$

---

$$
\begin{aligned}
    P(x) &= 256 x^8-512 x^6+304 x^4-48 x^2+1\\
\end{aligned}
$$

$$
\begin{aligned}
    \cos 9°   &= +\frac{1}{8} \left(2\sqrt{5-\sqrt{5}}+\sqrt{10}+\sqrt{2}\right)\\
    \cos 27°  &= +\frac{1}{8} \left(2\sqrt{5+\sqrt{5}}+\sqrt{10}-\sqrt{2}\right)\\
    \cos 63°  &= +\frac{1}{8} \left(2\sqrt{5+\sqrt{5}}-\sqrt{10}+\sqrt{2}\right)\\
    \cos 81°  &= -\frac{1}{8} \left(2\sqrt{5-\sqrt{5}}-\sqrt{10}-\sqrt{2}\right)\\
    \cos 99°  &= +\frac{1}{8} \left(2\sqrt{5-\sqrt{5}}-\sqrt{10}-\sqrt{2}\right)\\
    \cos 117° &= -\frac{1}{8} \left(2\sqrt{5+\sqrt{5}}-\sqrt{10}+\sqrt{2}\right)\\
    \cos 153° &= -\frac{1}{8} \left(2\sqrt{5+\sqrt{5}}+\sqrt{10}-\sqrt{2}\right)\\
    \cos 171° &= -\frac{1}{8} \left(2\sqrt{5-\sqrt{5}}+\sqrt{10}+\sqrt{2}\right)\\
\end{aligned}
$$